\zheader{Homeomorfismo inducido por una identificación}{suprayectiva, identificación, homeomorfismo}{2d11}

\begin{proposition}
Sea $f : X \longrightarrow Y$ una identificación y suprayectiva. Si se define en $X$ la relación de equivalencia $x_1 \sim x_2$ si y sólo si $f(x_1) = f(x_2)$, entonces $X/\sim$ es homeomorfo a $Y$.
\end{proposition}

\begin{proof}
La relación definida es una relación de equivalencia, para cualquier función $f$. Defínase $\widetilde{f} : X/\sim \longrightarrow Y$ como $\widetilde{f}([x]) = f(x)$. Se tiene que $f$ está bien definida, pues si $[x_1] = [x_2]$ entonces $x_1 \sim x_2$, luego $f(x_1) = f(x_2)$ por definición de $\sim$, es decir, $\widetilde{f}([x_1]) = \widetilde{f}([x_2])$. Nótese que $\widetilde{f} \circ p = f$, donde $p : X \longrightarrow X/\sim$ es la aplicación cociente. Se tiene que
\begin{enumerate}[label=\textnormal{(\roman*)}]
\item $\widetilde{f}$ es suprayectiva, pues dado $y \in Y$, existe $x \in X$ tal que $y = f(x)$ por suprayectividad de $x$, luego $[x] \in X/\sim$ es tal que $\widetilde{f}([x]) = f(x) = y$,
\item $\widetilde{f}$ es inyectiva, pues si $[x_1], [x_2] \in X/\sim$ son tales que $\widetilde{f}([x_1]) = \widetilde{f}([x_2])$, entonces $f(x_1) = f(x_2)$, luego $x_1 \sim x_2$ y por tanto $[x_1] = [x_2]$.
\end{enumerate}

Existe pues la función inversa $\widetilde{f}^{-1}$. Como $f = \widetilde{f} \circ p$ es continua y $p$ es identificación, la propiedad universal de las identificaciones implica que $\widetilde{f}$ es continua. Además, dado que $p = \widetilde{f}^{-1} \circ  f$ es continua y $f$ es identificación, entonces $\widetilde{f}^{-1}$ también debe ser continua. Luego $\widetilde{f}$ es un homeomorfismo.
\end{proof}
