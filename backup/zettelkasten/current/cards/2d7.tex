\zheader{Identificación es casi homeomorfismo}{homeomorfismo, identificación}{2d7}

\begin{proposition}
Sea $f : X \longrightarrow Y$ una función biyectiva. Entonces $f$ es identificación si y sólo si $f$ es homeomorfismo.
\end{proposition}

\begin{proof}
Supongamos que $f$ es identificación. Si $U$ es abierto en $X$, entonces $f^{-1}(f(U)) = U$ es abierto en $X$, luego $f(U)$ debe de ser abierto en $Y$ por ser $f$ identificación. Luego $f$ es una función abierta y como es continua y biyectiva, por \hyperref[card:2b1]{\textsf{2b1}} $f$ debe ser homeomorfismo. Recíprocamente, si $f$ es homeomorfismo, entonces $f$ es abierta nuevamente por \hyperref[card:2b1]{\textsf{2b1}} y como $f$ es continua y suprayectiva, entonces $f$ es identificación por \hyperref[card:2d5]{\textsf{2d5}}.
\end{proof}
