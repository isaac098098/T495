\zheader{Caracterización de identificaciones}{caracterización, compatibilidad, identificación}{2c12}

\begin{definition}
Dada una función $f : X \longrightarrow Y$, se dice que $g : X \longrightarrow Z$ es compatible con $f$ si $f(x_1) = f(x_2)$ implica que $g(x_1) = g(x_2)$, para cada $x, x' \in X$. 
\end{definition}

\begin{theorem}
Sea $f : X \longrightarrow Y$ continua y suprayectiva. Entonces $f$ es identificación si y sólo si para cada función continua $g : X \longrightarrow Z$ compatible con $f$, existe una única función continua $\overline{g} : Y \longrightarrow Z$ tal que $\overline{g} \circ f = g$.
\bigskip

\adjustbox{scale=1.2,center}{
    \begin{tikzcd}[row sep=large,column sep=large]
        X \arrow[swap]{d}{f} \arrow{r}{g} & Z \\
        Y \arrow[swap,dashed]{ru}{\overline{g}}
    \end{tikzcd}
}
\bigskip

Se dice que $\overline{g}$ \itshape{es el resultado de pasar $g$ al cociente.}

\end{theorem}

\begin{proof}
Supóngase que $f$ es identificación y sea $g : X \longrightarrow Z$ una función continua compatible con $f$. Defínase $\overline{g} : Y \longrightarrow Z$ como $\overline{g}(y) = g(x)$, donde $x \in X$ es tal que $y = f(x)$. Se tiene que $\overline{g}$ está bien definida, pues si $y_1, y_2 \in Y$ son tales que $y_1 = y_2$, entonces existen $x_1, x_2 \in X$ tales que $y_1 = f(x_1)$ y $y_2 = f(x_2)$, luego $f(x_1) = f(x_2)$ y por tanto $g(x_1) = g(x_2)$ por la compatibilidad de $g$ con $f$, es decir, $\overline{g}(y_1) = \overline{g}(y_2)$. Nótese que $\overline{g} \circ f = g$.
\bigskip

Si $\overline{g}' : Y \longrightarrow Z$ es una función tal que $\overline{g}' \circ f = g$. Si $y_1 = y_2$, entonces existen $x_1, x_2 \in X$ tales que $y_1 = f(x_1)$ y $y_2 = f(x_2)$, luego $f(x_1) = f(x_2)$ y por tanto $g(x_1) = g(x_2)$, luego $\overline{g}'(y_1) = \overline{g}'(f(x_1)) = g(x_1) = g(x_2) = \overline{g}(f(x_2)) = \overline{g}(y_2)$. En consecuencia, $\overline{g}' = \overline{g}$ y por lo tanto $g$ es la única función bajo las hipótesis con esta propiedad. Además, por hipótesis $g$ es continua y $f$ es identificación, luego la propiedad universal de las identificaciones implica que $\overline{g}$ debe ser continua. Más aún, si $g$ es identificación, como $f$ es identificación, \hyperref[card:2c3]{\textsf{2c3}} implica que $\overline{g}$ también es identificación.
\bigskip

Supóngase ahora que se verifica la condición. Defínase en $X$ la relación de equivalencia $x_1 \sim x_2$ si y sólo si $f(x_1) \sim f(x_2)$ y sea $p : X \longrightarrow X/\sim$ la aplicación cociente. Si $f(x_1) = f(x_2)$ entonces $x_1 \sim x_2$ y por tanto $p(x_1) = p(x_2)$, en consecuencia $p$ es compatible con $f$ y como también $p$ es continua, por hipótesis debe existir una función continua $\overline{p} : Y \longrightarrow X/\sim$ tal que $p = \overline{p} \circ f$. Por otro lado, nótese que si $p(x_1) = p(x_2)$, entonces $x_1 \sim x_2$ y por tanto $f(x_1) = f(x_2)$, es decir, $f$ es compatible con $p$. Como $p$ es identificación, entonces la primera parte de la demostración implica que existe una función continua $\overline{f} : X/\sim \longrightarrow Y$ tal que $f = \overline{f} \circ p$.
\bigskip

Se tiene entonces que $p = \overline{p} \circ \overline{f} \circ p$ y $f = \overline{f} \circ \overline{p} \circ f$. Afirmamos que $\overline{f} \circ \overline{p} = \text{id}_Y$. En efecto, si $y \in Y$, entonces existe $x \in X$ tal que $y = f(x)$, luego $y = f(x) = \overline{f}(\overline{p}(f(x))) = \overline{f}(\overline{p}(y))$. Como $y$ es arbitrario esto prueba la afirmación. Similarmente se prueba que $\overline{p} \circ \overline{f} = \text{id}_{X/\sim}$. Se tiene pues que $\overline{f}$ es un homeomorfismo y por tanto identificación, y dado que $p$ también es identificación y $f = \overline{f} \circ p$, entonces $f$ es identificación.
\end{proof}
