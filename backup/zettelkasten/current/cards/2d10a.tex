\zheader{Propiedades de saturación}{definición, saturación, identificación}{2d10a}

\begin{definition}
Si $p : X \longrightarrow X/\sim$ es una aplicación cociente y $A \subset X$, se define la \itshape{saturación} de $A$ como el conjunto $p^{-1}(p(A))$, que contiene a todos los puntos de $A$ y a todos los puntos en $X$ equivalentes a algún punto de $A$. Se dice que $A$ es \itshape{saturado} si $A = p^{-1}(p(A))$.
\end{definition}

\begin{proposition}
Sea $A \subset X$ un conjunto saturado respecto a una relaión de equivalencia $\sim$ y sea $p$ la respectiva aplicación cociente. Se tiene que
\begin{enumerate}[label=\textnormal{(\roman*)}]
\item Si $A \subset X$ es abierto o cerrado, entonces $p|_A : A \longrightarrow p(A)$ es una identificación.
\item Si $p$ es abierta o cerrada, entonces $p|_A : A \longrightarrow p(A)$ es una identificación.
\end{enumerate}
\end{proposition}

\begin{proof}
({\romannumeral 1}) Como $A$ es saturado, entonces $A = p^{-1}(p(A))$ y dado que $p$ es identificación y $A$ es abierto, $p(A)$ debe ser abierto en $X/\sim$. Y nuevamente, como $A = p^{-1}(p(A))$, entonces $p|_A : A \longrightarrow p(A)$ es una identificación por \hyperref[card:2d8]{\textsf{2d8}}.
\bigskip

({\romannumeral 2}) Sea $U$ un abierto en $A$. Entonces $U = V \cap A$, para algún abierto $V$ de $X$. Se tiene que $p(V \cap A) = p(V) \cap p(A)$. En efecto, en general se sabe que $p(V \cap A) \subset p(V) \cap p(A)$. Si $y \in p(V) \cap p(A)$, entonces existen $v \in V$ y $a \in A$ tales que $ p(v) = y = p(a)$, luego $p(a) \in p(A)$ y por tanto $p(v) \in p(A)$, luego $v \in p^{-1}(p(A)) = A$, por ser $A$ saturado. En consecuencia, $v \in V \cap A$ y por tanto $y = p(v) \in p(V \cap A)$. Esto prueba la afirmación. Luego, $p|_A(U) = p|_A(V \cap A) = p(V \cap A) = p(V) \cap p(A)$, donde $p(V)$ es abierto por ser $p$ una función abierta, así que $p|_A(U)$ es abierto en $p(A)$. Se sigue que $p|_A$ es también una función abierta y además es continua y suprayectiva. En consecuencia, $p|_A$ es una identificación.
\end{proof}
