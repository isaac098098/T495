\zheader{Definición de topología de identificación}{generación, definición}{2d}

\begin{definition}
Dados un espacio topológico $X$, un conjunto $Y$ y una función $f : X \longrightarrow Y$, se puede dotar a $Y$ con una topología, a saber, $\{ U \subset Y \mid f^{-1}(U) \text{ es abierto en } X \}$. A esta topología se le llamará \itshape{topología de identificación} o \itshape{topología coinducida} en $Y$ por $X$ a través de $f$.
\end{definition}

\begin{definition}
Si $X$ y $Y$ son espacios topológicos y $f : X \longrightarrow Y$ es una función, se dice que $f$ es una \itshape{identificación} si la topología de $Y$ es la topología coinducida por $f$.
\end{definition}
