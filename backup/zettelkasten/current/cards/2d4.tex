\zheader{Criterio para identificaciones}{criterio, sección}{2d4}

\begin{theorem}
Sea $p : X \longrightarrow Y$ continua. Si existe una función continua $s : Y \longrightarrow X$ tal que $p \circ s = \textnormal{id}_Y$, entonces $p$ es una identificación.
\end{theorem}

\begin{proof}
Si $U \subset Y$ es tal que $p^{-1}(U)$ es abierto en $X$, entonces 
\begin{equation*}
    s^{-1}(p^{-1}(U)) = (p \circ s)^{-1}(U) = \text{id}_Y(U) = U
\end{equation*}
es abierto, por ser $s$ continua. Como $p$ es también continua por hipótesis, se tiene que $U$ es abierto en $Y$ si y sólo si $p^{-1}(U)$ es abierto en $X$, luego $p$ es identificación.
\end{proof}

\begin{definition}
A $s : Y \longrightarrow X$ en el teorema anterior se le llama \itshape{sección} de $p$.
\end{definition}
