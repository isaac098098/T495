\zheader{Restricción continua e inyectiva}{contraejemplo, restricción, homeomorfismo}{2a2b}

\begin{remark}
En general, si $f : X \longrightarrow Y$ es continua e inyectiva, su restricción $g : X \longrightarrow f(X)$, dada por $g(x) = f(x), \forall x \in X$, no es necesariamente un homeomorfismo, aún cuando se tiene que $g$ es biyectiva y continua. Considere los espacios $X = \{ 0,1 \}$ con la topología $\mathcal{T}_X = \{ \emptyset, \{ 0 \}, X \}$ y $Y = \{ a, b, c \}$ con la topología $\mathcal{T}_Y = \{ \emptyset, \{ b \}, Y \}$. La función $f$ definida como $f(0) = c$,$f(1) = a$, es inyectiva y continua. Sin embargo, su restricción $g$, dada como $g(0) = a$, $g(1) = c$, es biyectiva y continua, pero su inversa $h$ dada por $h(a) = 0$, $h(c) = 1$ no es continua, pues $h^{-1}(\{ 0 \}) = \{ a \}$ no es abierto en $f(X)$ con la topología inducida por $Y$.
\end{remark}
