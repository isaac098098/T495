%%% Funciones que preservan relación
%% tags: preserva relación, continuidad

\zheader{Funciones que preservan relación}{2d12b}{preserva relación, continuidad}

\begin{definition}
Si $R$ y $S$ son relaciones en dos conjuntos $X$ y $Y$, respectivamente, se dice que $f$ \textbf{preserva relaciones} si $x \, R \, x'$ implica que $f(x) \, S \, f(x'), \forall x \in X$.
\end{definition}

\begin{corollary}
Sean $R$ y $S$ relaciones en dos espacios $X$ y $Y$, respectivamente y sean $p_X$, $p_Y$ las respectivas aplicaciones cociente. Si $f : X \longrightarrow Y$ es continua y preserva relaciones, entonces existe una única función continua $f_* : X/R \longrightarrow Y/S$ tal que $f_* \circ p_X = p_Y \circ f$. Es decir, tal que el siguiente diagrama conmuta 
\bigskip

\adjustbox{scale=1.2,center}{
    \begin{tikzcd}[row sep=large,column sep=large]
        X \arrow[swap]{d}{p_X} \arrow{r}{f} & Y \arrow{d}{p_Y} \\
        X/R \arrow[swap]{r}{f_*} & Y/S.
    \end{tikzcd}
}
\end{corollary}

\begin{proof}
Se sigue de \hyperref[card:2d12]{\textsf{2d12}} tomando $g = p_Y \circ f$.
\end{proof}
