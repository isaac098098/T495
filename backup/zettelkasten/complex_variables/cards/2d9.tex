%%% Propiedad universal de las identificaciones
%% tags: propiedad universal, identificación

\zheader{Propiedad universal de las identificaciones}{2d9}{propiedad universal, identificación}

\begin{theorem}
Sea $f : X \longrightarrow Y$ una función. Entonces $f$ es identificación si y sólo si se cumplen las siguientes condiciones:
\begin{enumerate}[label=\textnormal{(\roman*)}]
\item $f$ es continua.
\item Una función $g : Y \longrightarrow Z$ es continua si y sólo si $g \circ f$ es continua.
\end{enumerate}
\bigskip

\adjustbox{scale=1.2,center}{
    \begin{tikzcd}[row sep=large,column sep=large]
        X \arrow[swap]{d}{f} \arrow{dr}{g \circ f} & \\
        Y \arrow[swap,dashed]{r}{g} & Z
    \end{tikzcd}
}
\end{theorem}

\begin{proof}
Supóngase primero que $f$ es identificación. Entonces $f$ es continua y se tiene ({\scshape\romannumeral 1}). Sea $g : Y \longrightarrow Z$ una función. Si $g$ es continua, entonces $g \circ f$ es continua por ser composición de funciones continuas. Si $g \circ f$ es continua y $U$ es un abierto en $Z$, se tiene que $(g \circ f)^{-1}(U) = f^{-1}(g^{-1}(U))$ es abierto en $X$ y por tanto $g^{-1}(U)$ es abierto en $Y$ por ser $f$ identificación. Como $U$ fue arbitrario, entonces $g$ es continua y hemos probado ({\scshape\romannumeral 2}).
\bigskip

Suponga ahora que se verifican las condiciones y sean $\mathcal{T}$ la topología en $Y$ y $\mathcal{T}_f$ la topología coinducida por $f$ en $Y$. Defínase $f' : X \longrightarrow (Y,\mathcal{T}_f)$ como $f'(x) = f(x), \forall x \in X$. Se tiene que $f'$ es continua, pues si $U$ es abierto en $(Y,\mathcal{T}_f)$, entonces $f'^{-1}(U) = f^{-1}(U)$, el cual es abierto en $X$, pues $\mathcal{T}_f$ hace continua a $f$. Más aún, se tiene que $f' = \text{id}_Y \circ f$, donde $\text{id}_Y : (Y,\mathcal{T}) \longrightarrow (Y,\mathcal{T}_f)$, luego la condición ({\scshape\romannumeral 2}) implica que $\text{id}_Y$ es continua, así que $\mathcal{T}_f \subset \mathcal{T}$. Además, por la condición ({\scshape\romannumeral 1}), la topología $\mathcal{T}$ hace continua a $f$ y en consecuencia $\mathcal{T} \subset \mathcal{T}_f$. Se sigue que $\mathcal{T} = \mathcal{T}_f$, es decir, $f$ es una identificación.
\end{proof}
