%%% Proyección estereográfica
%% tags: homeomorfismo explícito, proyección estereográfica

\zheader{Proyección estereográfica}{2a2a}{homeomorfismo explícito, proyección estereográfica}

\begin{theorem}
La función
\begin{align*}
    p : S^n-\{ N \} & \longrightarrow \mathbb{R}^n \\
    (x_1,\ldots,x_{n+1}) & \longmapsto \left( \frac{x_1}{1-x_{n+1}}, \ldots, \frac{x_n}{1-x_{n+1}} \right),
\end{align*}
donde $N=(0,\ldots,0,1)$, es un homeomorfismo con las topologías usuales y su inversa está dada por
\begin{align*}
    p^{-1} : \mathbb{R}^n & \longrightarrow S^n - \{ N \} \\
    y = (y_1,\ldots,y_n) & \longmapsto \left( \frac{2 y_1}{|y|^2+1}, \ldots, \frac{2 y_n}{|y|^2+1}, \frac{|y|^2-1}{|y|^2+1} \right).
\end{align*}
A este homeomorfismo se le llama \textbf{proyección estereográfica}.
\end{theorem}

\begin{proof}
Es rutinario verificar que $p \circ p^{-1} = \text{id}_{\mathbb{R}^n}$ y que $p^{-1} \circ p = \text{id}_{S^n - \{ N \}}$. Además, $p$ es continua por ser sus componentes funciones racionales en las variables $x_1,\ldots,x_{n+1}$ tales que su denominador no se anula. De forma similar, $p^{-1}$ es continua por ser sus funciones componentes productos de las variables $y_1,\ldots,y_n$, con la función $1/(|y|^2+1)$, la cuál es continua pues el denominador no se anula y la función norma $|y|$ es continua.
\end{proof}
