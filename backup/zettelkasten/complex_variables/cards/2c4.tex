%%% Propiedades de abiertos y cerrados
%% tags: propiedades, abiertos, suma topológica

\zheader{Propiedades de abiertos y cerrados}{2c4}{propiedades, abiertos, suma topológica}

\begin{theorem}
Si $\{ X_{\lambda} \}_{\lambda}$ es una familia de espacios topológicos y $X = \coprod_{\lambda \in \Lambda} X_{\lambda}$ es su suma topológica, entonces
\begin{enumerate}[label=\textnormal{(\roman*)}, align=left, labelwidth=1.7em, leftmargin=3.0em]
\item Si $\mu \in \Lambda$, entonces $U$ es abierto en $X_{\mu}$ si y solo si $U \times \{ \mu \}$ es abierto en $X_{\mu} \times \{ \mu \}$.
\item Cada subespacio $X_{\lambda} \times \{ \lambda \}$ de $X$ es abierto y cerrado en $X$,
\item Si $\mu \in \Lambda$ y $: \subset X_{\mu}$, entonces $U$ es abierto en $X_{\mu}$ si y solo si $U \times \{ \mu \}$ es abierto en $X$,
\item $i_{\lambda} : X_{\lambda} \longrightarrow X$ es una función abierta, $\forall \lambda \in \Lambda$.
\item $i_{\lambda} : X_{\lambda} \longrightarrow X_{\lambda} \times \{ \lambda \}$ es una función abierta, $\forall \lambda \in \Lambda$.
\item $i_{\lambda} : X_{\lambda} \longrightarrow X_{\lambda} \times \{ \lambda \}$ es un homeomorfismo.
\end{enumerate}
\end{theorem}

\begin{proof}
({\scshape\romannumeral 1}) Si $\mu \in \Lambda$ y $U$ es abierto en $X_{\mu}$, entonces $U \subset X_{\mu}$ y por ({\scshape\romannumeral 1}) se tiene que $i^{-1}_{\mu}(U \times \{ \mu \}) = U$. Luego $U \times \{ \mu \}$ debe ser abierto en la topología coinducida en $X$ por $X_{\mu}$ a través de $i_{\mu}$, es decir, $U \times \{ \mu \} \in \mathcal{T}_{\mu}$. Más aún, si $\lambda \ne \mu$, por ({\scshape\romannumeral 2}) se tiene que $i^{-1}_{\lambda}(U \times \{ \mu \}) = \emptyset$, el cual también es abierto en $X_{\lambda}$. En consecuencia, $U \times \{ \mu \} \in \mathcal{T}_{\lambda}$, para cada $\lambda \in \Lambda$. Luego $U \times \{ \mu \} \in \bigcap_{\lambda \in \Lambda} \mathcal{T}_{\lambda} = \mathcal{S}$, es decir, $U \times \{ \mu \}$ es abierto en $X$.
\bigskip

Supóngase ahora que $U \times \{ \mu \}$ es abierto en $X_{\mu} \times \{ \mu \}$. Como la inclusión $i_{\mu} : X_{\mu} \longrightarrow X$ es continua y $i_{\mu}(X_{\mu}) = X_{\mu} \times \{ \mu \}$, entonces $i_{\mu} : X_{\mu} \longrightarrow X_{\mu} \times \{ \mu \}$ también es continua. Como $U \subset X_{\mu}$, entonces $i^{-1}_{\mu}(U \times \{ \mu \}) = U$ por ({\scshape\romannumeral 1}) y se sigue que $U$ es abierto en $X_{\mu}$.
\bigskip

({\scshape\romannumeral 2}) Corolario de \hyperref[card:2d3]{\textsf{2d3}}. ({\scshape\romannumeral 3}) Se sigue de ({\scshape\romannumeral 1}) y ({\scshape\romannumeral 2}). ({\scshape\romannumeral 4}) Se sigue de \hyperref[card:2d2]{\textsf{2d2}} ({\scshape\romannumeral 3}) y del punto anterior. ({\scshape\romannumeral 5}). Se sigue del punto anterior. ({\scshape\romannumeral 6}) Es fácil ver que $i_{\lambda} : X_{\lambda} \longrightarrow X_{\lambda} \times \{ \lambda \}$ es biyectiva. Es además continua y abierta, luego un homeomorfismo.

\end{proof}
