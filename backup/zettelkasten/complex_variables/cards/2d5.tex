%%% Criterio para identificaciones
%% tags: criterio, abierta, identificación

\zheader{Criterio para identificaciones}{2d5}{criterio, abierta, identificación}

\begin{proposition}
Si $f : X \longrightarrow Y$ es continua, suprayectiva y abierta o cerrada, entonces $f$ es identificación.
\end{proposition}

\begin{proof}
Si $U \subset Y$ es tal que $f^{-1}(U)$ es abierto en $X$, entonces $U = f(f^{-1}(U))$ debe ser abierto en $Y$ por ser $f$ suprayectiva y abierta. Como $f$ también es continua, entonces $f$ debe ser identificación por \hyperref[card:2d2]{\textsf{2d2}}. Si $f$ es cerrada, la demostración es similar usando nuevamente \hyperref[card:2d2]{\textsf{2d2}}.

\end{proof}
