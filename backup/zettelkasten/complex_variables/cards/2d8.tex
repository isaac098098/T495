%%% Restricción de identificaciones
%% tags: restricción, identificación, criterio

\zheader{Restricción de identificaciones}{2d8}{restricción, identificación, criterio}

\begin{theorem}
Si $f : X \longrightarrow Y$ es identificación, $B$ es abierto o cerrado en $Y$ y $A = f^{-1}(B)$, entonces $f|_{A} : A \longrightarrow B$ es identificación.
\end{theorem}

\begin{proof}
Como $f$ es continua, entonces $f|_A : A \longrightarrow Y$ es continua. Más aún, como $B \subset Y$ y $f(A) = f(f^{-1}(B)) \subset B$, entonces $f|_A : A \longrightarrow B$ es continua. Sea $U \subset B$ tal que $f|_A^{-1}(U)$ es abierto en $A$. Como $B$ es abierto en $Y$, entonces $f^{-1}(B) = A$ es abierto en $X$, por ser $f$ continua y por tanto $f|_A^{-1}(U)$ es abierto en $X$. Pero
\begin{equation*}
    f|_A^{-1}(U) = f^{-1}(U) \cap A = f^{-1}(U) \cap f^{-1}(B) = f^{-1}(U \cap B) = f^{-1}(U),
\end{equation*}
por ser $U \subset B$, así que $f^{-1}(U)$ es abierto en $X$. Como $f$ es identificación, esto implica que $U$ es abierto en $Y$ y por tanto $U$ es también abierto en $B$, pues $U = U \cap B$. Como $U$ fue arbitrario, entonces $f|_A$ es identificación. Si $B$ es cerrado la demostración es similar.
\end{proof}
