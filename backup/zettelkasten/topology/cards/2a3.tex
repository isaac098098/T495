%%% Lema del pegado
%% tags: cerrados, continuidad, lema del pegado

\zheader{Lema del pegado}{2a3}{cerrados, continuidad, lema del pegado}

\begin{theorem}
Sea $X = F_1 \cup \cdots \cup F_k$, con $F_i$ cerrado en $X$, para todo $i \in \{ 1,\ldots,k \}$. Si $f_i : F_i \longrightarrow Y$ son funciones continuas, para todo $i \in \{ 1,\ldots,k \}$ y tales que
\begin{equation*}
    f_i|_{F_i \cap F_j} = f_j|_{F_i \cap F_j},
\end{equation*}
para todos $i,j \in \{ 1,\ldots,k \}$, entonces la función $f : X \longrightarrow Y$ definida como $f|_{F_i} = f_i$ es continua.
\end{theorem}

\begin{proof}
Dicha función está bien definida, pues si $x_1, x_2 \in X$ son tales que $x_1 = x_2$, entonces $x_1 \in F_i$ y $x_2 \in F_j$ para algunos $i,j \in \{ 1,\ldots,k \}$ y $x_1 \in F_i \cap F_j$, $x_2 \in F_i \cap F_j$. Luego $f(x_1) = f_i(x_1)$ y $f(x_2) = f_j(x_2)$, por tanto, $f(x_1) = f_i(x_1) = f_i|_{F_i \cap F_j}(x_1) = f_i|_{F_i \cap F_j}(x_2) = f_j|_{F_i \cap F_j} (x_2) = f_j(x_2) = f(x_2)$.
\bigskip

Sea $C$ un cerrado en $Y$. Se tiene que $f^{-1}(X) = f_1^{-1}(C) \cup \cdots \cup f_k^{-1}(C)$. En efecto, si $x \in f^{-1}(C)$, entonces $f(x) \in C$, con $x \in X$. Luego, dado que $X = \bigcup_{n=1}^{k} F_n$, existe $i \in \{ 1,\ldots,k \}$ tal que $x \in F_i$ y por tanto $f(x) = f_i(x) \in C$. En consecuencia, $x \in f^{-1}_i (C)$ para algún $i \in \{ 1,\ldots,k \}$, es decir, $x \in \bigcup_{n=1}^{k} f^{-1}_n(C)$. Recíprocamente, si $x \in \bigcup_{n=1}^{k} f^{-1}_n(C)$, entonces existe $i \in \{ 1,\ldots,k \}$ tal que $x \in f^{-1}_i(C)$, por tanto, $f_i(x) \in C$. Necesariamente $x \in F_i$ por elección de $x$, así que $f_i(x) = f(x) \in C$, o bien, $x \in f^{-1}(C)$. Esto prueba la afirmación.
\bigskip

Finalmente, como $f_i$ es continua, para cada $i \in \{ 1,\ldots,k \}$, entonces $f^{-1}_i(C)$ es cerrado en $F_i$, para cada $i \in \{ 1,\ldots,k \}$, pero cada $F_i$ es cerrado en $X$, así que de hecho $f^{-1}_i(C)$ es cerrado en $X$, para cada $i \in \{ 1,\ldots,k \}$. Luego $f^{-1}(C)$ es cerrado en $X$ por ser unión finita de cerrados en $X$. Como $C$ fue arbitrario, entonces $f$ debe ser continua.
\end{proof}
