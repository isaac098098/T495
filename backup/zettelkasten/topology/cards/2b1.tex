\zheader{Criterio para homeomorfismos}{criterio, abierta, cerrada}{2b1}

\begin{theorem}
Si una función $f : X \longrightarrow Y$ es biyectiva, continua y abierta o cerrada, entonces $f$ es un homeomorfismo.
\end{theorem}

\begin{proof}
Como $f$ es biyectiva, existe su inversa $g : Y \longrightarrow X$ tal que $g \circ f = \text{id}_X$. Sea $U$ un abierto en $X$ y notemos que $f^{-1}(g^{-1}(U)) = (g \circ f)^{-1}(U) = \text{id}_X^{-1}(U) = U$ y aplicando $f$ a ambos lados obtenemos $g^{-1}(U) = f(U)$ por suprayectividad de $f$. Como $f(U)$ es abierto por ser $f$ una funcion abierta, entonces $g^{-1}(U)$ es abierto. Dado que $U$ fue un abierto arbitrario, entones $g$ es continua y en consecuencia $f$ es un homeomorfismo. Si $f$ es cerrada la demostración es similar.
\end{proof}

\begin{theorem}
Si $f$ es un homeomorfismo, entonces $f$ es abierta y cerrada.
\end{theorem}

\begin{proof}
Sea $g$ la inversa de $f$. Si $U$ es abierto en $X$, entonces $g^{-1}(U)$ es abierto en $X$ por ser $g$ continua, pero, de manera similar al teorema anterior, se tiene que $g^{-1}(U) = f(U)$, luego $f(U)$ es abierto y se sigue que $f$ es una función abierta. Similarmente se prueba que $f$ es cerrada.
\end{proof}
