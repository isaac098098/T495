%%% Propiedades de las identificaciones
%% tags: propiedades, composición, identificación

\zheader{Propiedades de las identificaciones}{2d3}{propiedades, composición, identificación}

\begin{proposition}
Sean $f : X \longrightarrow Y$ y $g : Y \longrightarrow Z$ funciones. Se verifican las siguientes afirmaciones
\begin{enumerate}[label=\textnormal{(\roman*)}]
\item $\textnormal{id}_X : X \longrightarrow X$ es identificación.
\item Si $f$ y $g$ son identificaciones, entonces $g \circ f$ es identificación.
\item Si $f$ y $g \circ f$ son identificaciones, necesariamente $g$ es identificación.
\end{enumerate}
\end{proposition}

\begin{proof}
({\scshape\romannumeral 1}) Se sigue de que $U$ es abierto en $X$ si y sólo si $\text{id}_X(U) = U$ es abierto en $X$.
\bigskip

({\scshape\romannumeral 2}) Como $f$ y $g$ son identificaciones, entonces, por \hyperref[card:2d2]{\textsf{2d2}},
\begin{align*}
    U \text{ es abierto en } Z & \iff g^{-1}(U) \text{ es abierto en } Y \\
                               & \iff f^{-1}(g^{-1}(U)) = (g \circ f)^{-1}(U) \text{ es abierto en } X.
\end{align*}
Luego $g \circ f$ es identificación.
\bigskip

({\scshape\romannumeral 3}) Se tiene que
\begin{align*}
    U \text{ es abierto en } Z & \iff (g \circ f)^{-1}(U) = f^{-1}(g^{-1}(U)) \text{ es abierto en } X \\
                               & \iff g^{-1}(U) \text{ es abierto en } Y,
\end{align*}
luego $g$ es identificación.
\end{proof}
