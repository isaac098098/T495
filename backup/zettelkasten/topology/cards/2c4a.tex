%%% Propiedades de una sección
%% tags: propiedades, sección

\zheader{Propiedades de una sección}{2c4a}{propiedades, sección}

\begin{theorem}
Si $s : Y \longrightarrow X$ es una sección de $p : X \longrightarrow Y$, entonces
\begin{enumerate}[label=\textnormal{(\roman*)}]
\item $s$ es inyectiva,
\item $s$ es un encaje, es decir, $Y \cong s(Y)$.
\end{enumerate}
\end{theorem}

\begin{proof}
({\scshape\romannumeral 1}) Si $y_1, y_2 \in Y$ son tales que $s(y_1) = s(y_2)$, entonces $p(s(y_1)) = p(s(y_2))$, pero $p \circ s = \text{id}_Y$, en consecuencia $y_1 = y_2$. Luego $s$ es inyectiva.
\bigskip

({\scshape\romannumeral 2}) Sea $r : Y \longrightarrow s(Y)$ la restricción de $s$ al contradominio $s(Y)$. Claro que $r$ es biyectiva, pues es suprayectiva por construcción e inyectiva por ser $s$ inyectiva. Más aún, $r$ es continua, pues $s$ es continua y $s(Y) \subset X$. Sea $U$ un abierto en $Y$. Como $p$ es continua, entonces $p^{-1}(U)$ debe ser abierto en $X$, además
\begin{align*}
    r^{-1}(p^{-1}(U) \cap s(Y)) &= s^{-1}(p^{-1}(U) \cap s(Y)) = s^{-1}(p^{-1}(U)) \cap s^{-1}(s(Y)) \\
                                &= (p \circ s)^{-1}(U) \cap Y, \text{ por inyectividad de } s \\
                                &= \text{id}^{-1}_Y(U) \cap Y \\
                                &= U \cap Y = U.
\end{align*}
Tomando la imagen bajo $r$ a ambos lados, se tiene que $p^{-1}(U) \cap s(Y) = r(U)$, por ser $r$ suprayectiva. Se sigue que $r(U)$ es un abierto en $s(Y)$. Como $U$ fue un abierto arbitrario de $Y$, entonces $r$ es una función abierta. Luego, como $r$ es biyectiva, continua y abierta, entonces $r$ es un homeomorfismo por \hyperref[card:2b1]{\textsf{2b1}} y por tanto $s$ es un encaje.
\end{proof}
