%%% Networking
\seclecture{Networking}{Thu 17 Jul 25}

\subsection{Algunas definiciones básicas}

La \textbf{Internet} es una red de computadoras que conecta a cientos de millones de dispositivos informáticos en todo el mundo. Estos dispositivos incluyen PC's, Linux workstations, servers, laptops, smartphones, tablets, TV's, consolas de videojuegos, Webcams, automóviles, dispositivos de detección ambiental o sistemas domésticos eléctricos y de seguridad. Cualquier dispositivo que se conecte a Internet se llamará \textbf{host} o \textbf{end system}, es decir, anfitrión o sistema final.
\bigskip

\marginpar{\footnotesize{Wed 23 Jul 25}}
Los sistemas finales se conectan entre sí con una red de \textbf{communication links} o enlaces de comunicación, y \textbf{packet switches} o conmutadores de paquetes. Diferentes enlaces pueden transmitir datos a diferentes tasas, siendo la \textbf{transmission rate} o tasa transmisión de un enlace medida en \texttt{bits/second}. Cuando un sistema final envía datos a otro sistema final, el sistema final que envía segmenta los datos y agrega \textbf{header bytes} o bytes de encabezado a cada segmento. Los paquetes de información resultantes, conocidos como \textbf{packets} o paquetes en la jerga de redes, se envían a través de la red hacia el sistema final de destino, donde son reensamblados en los datos originales.
\bigskip

Un conmutador de paquetes toma paquetes que llegan de alguno de sus enlaces de comunicación entrantes y reenvía el paquete a uno de sus enlaces de comunicación salientes. Existen varios tipos de conmutadores de paquetes, pero los más destacados hoy en día son los \textbf{routers} o enrutadores, y los \textbf{link-layer switches} o conmutadores de capa de enlace. Ambos conmutadores reenvían los paquetes hacia su destino final. Los conmutadores de capa de enlace generalmente se usan en \textbf{access networks} o redes de acceso, mientras que los enrutadores se suelen usar en el \textbf{network core} o el núcleo de red. La sucesión de enlaces de comunicación y conmutadores de paquetes atravesados por un paquete del sistema final que envía al sistema final que lo recibe, se llama \textbf{route} o \textbf{path}, o ruta o camino del paquete.
\bigskip

Los sistemas finales acceden a Internet a través de un \textbf{Internet Service Provider (ISP)}, incluyendo ISPs residenciales como compañías locales de cable o teléfono, ISPs corporativos, ISPs universitarios y ISPs que proveen acceso a WiFi en aeropuertos, hoteles, cafeterías y otros lugares públicos. Cada ISP es en sí misma una red de enlaces de comunicación y conmutadores de paquetes.
\bigskip

Los ISPs proporcionan una variedad de tipos de acceso a los sistemas finales, incluyendo \textit{residential broadband access} o acceso banda ancha residencial como un módem de cable o Digital Subscriber Line (DSL), \textit{high-speed local area network access} o acceso a red de área local de alta velocidad, \textit{wireless access} o acceso inalámbrico, o \textit{56kbps dial-up modem access} o acceso a módem telefónico de 56kbps. Los ISPs también proveen acceso a Internet a proveedores de contenido, conectando los sitios Web directamente a Internet. Los ISPs también están conectados entre sí para dar acceso a más sistemas finales. Estos \textbf{lower-tier ISPs} o ISPs de nivel inferior están conectados a través de \textbf{upper-tier ISPs} o ISPs de nivel superior nacionales e internacionales como Level 3 Communications, AT\&T, Sprint y NTT. Cada red ISP se gestiona de forma independiente, usa el protocolo IP y se adhiere a ciertas convenciones de nombres y direcciones.
\bigskip

\marginpar{\raggedright\footnotesize{descargar estos documentos y agregarlos a la bibliografía para estudiarlos en el futuro, cuando ya puedas entenderlos}}
Los sistemas finales, conmutadores de paquetes y otras piezas de Internet ejecutan protocolos que controlan el envío y recibo de información dentro de Internet. El \textbf{Transmission Control Protocol (TCP)} y el \textbf{Internet Protocol (IP)} son dos de los más importantes. El protocolo IP especifica el formato de los paquetes que se envían y reciben entre los enrutadores y los sistemas finales. Los protocolos principales de Internet se conocen colectivamente como \textbf{TCP/IP}. Para estableces estándares de protocolos se crearon los \textbf{Internet standards} y son desarrollados por la Internet Engineering Task Force (IETF). Los documentos estándar de la IETF se llaman \textbf{request for comments (RFCs)} y entre estos están las definiciones de los protocolos TCP, IP, HTTP (para la Web) y SMTP (para e-mail). Para especificaciones de estándares de componentes de red y en particular enlaces de red, existen otros cuerpos de información. Por ejemplo, el IEEE 802 LAN/MAN Standards Committee especifica los estándares de Ethernet y wireless WiFi.
