%%% Operating Systems
\seclecture{Operating Systems}{Thu 17 Jul 25}

Existen tres conceptos clave para entender como funciona un sistema operativo, cómo decide qué programa ejecutar después en la CPU, cómo maneja la sobrecarga de memoria en un sistema de memoria virtual, cómo funcionan los monitores de máquinas virtuales, cómo manejar la información en discos o incluso cómo construir un sistema distribuido que funcione cuando algunas partes hayan fallado. Estos conceptos son los de \textbf{virtualización}, \textbf{concurrencia} y \textbf{persistencia}.

\subsection{Sistemas informáticos}

El código máquina \texttt{x86-64} es el último en un camino de evolución seguido por Intel y su competencia que empezó con el procesador \texttt{8086} en 1978. Esta clase de procesadores se conoce coloquialmente como \texttt{x86}. Consideraremos como las máquinas ejecutan código \texttt{C} en Linux. Otros sistemas operativos que tienen herencia del sistema operativo Unix son Solaris, FreeBSD y MacOS X. Estos sistemas han mantenido un buen nivel de compatibilidad gracias a Posix y al Single UNIX Specification. Se puede usar Linux en un entorno de máquina virtual como VirtualBox o VMWare, que permiten ejecutar programas escritos para un sistema operativo, el \textit{guest OS} en otro, el \textit{host OS}.
