%%% Introduction

\section{Complex Numbers}

\subsection{Definitions and Notation of Complex Numbers}

\subsubsection{A bit of history}

Luca Pacioli published in 1494 the book ``Summa de arithmetica, geometria, proportioni  et  proportionalita'',  in  this  text  the  general  cubic  equation $a x^3 + b x^2 + d = 0$  was proposed and it was believed that such an equation was impossible  to  solve,  however,  in  the  following  decades  and  centuries  some mathematicians  began  to  find  some  partial  solutions  of  the  general  cubic equation. 
\bigskip

Some mathematicians like  Niccolò Tartaglia and Scipione del Ferro solved the cubic equation in the 16th century, but partially. In that same century, Gerolamo Cardano published an algebraic method to analytically solve any cubic equation, Cardano solved the cubic equation, but, only the real ones and he was the first to introduce complex numbers $a + \sqrt{-b}$. 
\bigskip

Rafael Bombelli was an mathematician and engineer is central figure in the understanding of imaginary numbers. He was the one who finally managed to address  the  problem  with  imaginary  numbers.  In  his  1572  book,  L\textquotesingle Algebra, Bombelli  solved  equations  using  the  method  of  del Ferro/Tartaglia.  He introduced the rhetoric that preceded the representative symbols $\mathbf{+i}$ and $\mathbf{-i}$ and described  how  they  both  worked  because  they  were  necessary  for  his calculations. Bombelli introduces a notation for $\sqrt{-1}$, and calls it ``pí u di meno''. 
\bigskip

René  Descartes  associated  imaginary  numbers  with  geometric  impossibility. Euler, first of all, introduced the notation $i = \sqrt{-1}$ , and visualized complex numbers as points with rectangular coordinate while Gauss introduced the term complex number and Cauchy constructed the set of complex numbers in 1847.
\bigskip

Complex numbers are used in many fields of engineering and science, including:
\begin{itemize}
\item Electronics.
\item Electromagnetism.
\item To simplify the unknown roots if roots are not real for quadratic equations.
\item Computer science engineering.
\item Civil engineering.
\item Control systems.
\item Quantum mechanics, $H \Psi = i \hbar \partial \Psi / \partial t$.
\end{itemize}

\subsubsection{Introduction}

A complex number is an element of a number system that extends the real numbers  with  a  specific  element  denoted  $i$,  called  the  imaginary  unit  and satisfying the equation
\begin{equation}
    i^2 = -1
\end{equation}
or equivalently
\begin{equation}
    \sqrt{-1} = i.
\end{equation}

Every complex number can be expressed in the form $a + b i$, where $a$ and $b$ are real numbers. Because no real number satisfies the above equation, $i$ was called an imaginary number by René descartes. For the complex number $a + b i$,  $a$ is called the real  part and $b$ is  called the imaginary  part.  The set  of complex numbers  is  denoted  by  either  of  the  symbols $\mathbf{C}$ or $\mathbb{C}$. Despite  the  historical nomenclature ``imaginary'', complex numbers are regarded in the mathematical sciences as just as ``real'' as the real numbers and are fundamental in many aspects of the scientific description of the natural world. Complex  numbers  can  be  expressed  in  three  different  ways:  binomial, rectangular and exponencial form.

Complex  numbers  can  be  expressed  in  three  different  ways:  binomial, rectangular and exponencial form.

\subsection{Rectangular or binomial form of complex numbers and the complex plane}

It  is  customary  to  express  any  real  number  by  means  of  the  letter  $z$.  The rectangular form of a complex number is a sum of two terms: the number\textquotesingle s real part and the number\textquotesingle s imaginary part multiplied by $i$.
\begin{equation}
    z = a + b i
\end{equation}

It  is  possible also plot  a  complex  number given in  rectangular form in  the complex plane (Fig. 1.1). The real and imaginary parts determine the real and imaginary coordinates of the number.
