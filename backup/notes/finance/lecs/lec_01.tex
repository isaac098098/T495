%%% Probabilidad
\seclecture{Probabilidad}{Mon 23 Sep 24}
\subsection{Probabilidad de un evento}

Una operación en masa siempre consiste en la repetición de un gran número de operaciones idénticas e individuales. Nos interesan los resultados bien definidos de estas operaciones individuales y el número de dichos resultados en una operación en masa. El porcentaje, o en general, la parte fraccionaria de dichos resultados ``exitosos'' se llamará la \emph{probabilidad} de ese resultado. Sin embargo, en la teoría de la probabilidad se acostumbra a llamara a aquellos resultados que llevan a la realización del evento en el que estamos interesados resultados ``exitosos''. La cuestión de la probabilidad de un evento (resultado) solo tiene sentido bajo condiciones bien definidas bajo las cuáles la operación en masa tiene lugar, un cambio en estas condiciones podría cambiar la probabilidad del evento en consideración.
\bigskip
\marginnote{\footnotesize\textsf{\mbox{Tue 24 Sep 24}}}

Si la operación en masa es tal que un evento $A$ se observa en promedio $a$ veces en $b$ operaciones individuales, entonces la probabilidad del evento $A$ bajo las condiciones dadas es $a/b$ o $100 a/b\%$. La \emph{probabilidad} de un resultado ``exitoso'' de una operación individual es el \emph{cociente entre el número de dichos resultados ``exitosos'' y el número de operaciones individuales} que constituyen la operación en masa. En una colección de $b$ operaciones individuales, este evento puede ocurrir más de $a$ veces o menos de $a$ veces, es sólo en \emph{promedio} que ocurre aproximadamente $a$ veces. En la mayoría de dichas colecciones de eventos, el número de ocurrencias del evento $A$ será cercano a $a$, particulamente si $b$ es un número grande.

\subsection{Eventos imposibles y seguros}

La probabilidad de un evento es siempre positiva o cero y no más grande que uno. Se denota la probabilidad del evento $A$ como $P(A)$, y se tiene $0 \le P(A) \le 1$. Mientras más grande es $P(A)$, el evento ocurrirá más seguido. Si $P(A)=0$, entonces el evento nunca ocurre u ocurre muy rara vez y en la práctica se le puede considerar \emph{imposible}. Si $P(A)=1$, entonces el evento $A$ ocurre siempre o casi siempre y en la práctica se puede asumir que su ocurrencia es \emph{segura}. Si $P(A)>1/2$ el evento $A$ ocurre más frecuentemente de lo que no ocurre y si $P(A)<1/2$ se tiene el fenómeno contrario.
\bigskip

Qué tan pequeña debe ser la probabilidad de un evento para considerarlo imposible depende de que tan importante es el evento con el que estamos trabajando.

\section{Reglas para la adición de probabilidades}

\subsection{Derivación de la regla para la adición de probabilidaes}

Suponga que en la ejecución de un resultado en masa, se estableció que en cada serie de $b$ operaciones individuales en promedio cierto resultado $A_1$ se observó $a_1$ veces, otro resultado $A_2$ se observó $a_2$ veces, otro resultado $A_3$ fue observado $a_3$ veces y así sucesivamente, o en otras palabras, que la probabilidad del evento $A_1$ es igual a $a_1/b$, la del evento $A_2$ es igual a $a_2/b$, etc. ¿Qué tan grande es la probabilidad de que, en una operación individual, alguno de los resultados $A_1,A_2,A_3,\ldots$ ocurra? A este evento se le llama ``$A_1$ o $A_2$ o $A_3$ o $\ldots$''. En una serie de $b$ operaciones, \textit{este evento ocurre} $a_1+a_2+a_3+\cdots$ \textit{veces en promedio}; esto significa que la probabilidad buscada es igual a
\begin{equation*}
	\frac{a_1+a_2+a_3+\cdots}{b}=\frac{a_1}{b}+\frac{a_2}{b}+\frac{a_3}{b}+\cdots,
\end{equation*}
o bien
\begin{equation*}
    P(A_1 \text{ o } A_2 \text{ o } A_3 \text{ o } \ldots) = P(A_1)+P(A_2)+P(A_3)+\cdots
\end{equation*}

\marginnote{\footnotesize\textsf{\mbox{Fri 27 Sep 24}}}

Siempre asumiremos que cualesquiera dos de los resultados en consideración son \emph{mutuamente incompatibles} o \emph{mutuamente excluyentes}, es decir, que no se pueden observar juntos en la misma operación individual. Sin esta suposición la regla de adición de probabilidades anterior ya no es válida.

\subsection{Sistemas completos de eventos}

Dos eventos se dicen \emph{eventos complementarios} si uno y sólo uno de ellos (mutuamente excluyentes) ocurre siempre en una operación individual. Si $A_1$ y $A_2$ son eventos complementarios y en una serie de $b$ operaciones el evento $A_1$ ocurre $a_1$ veces y el evento $A_2$ ocurre $a_2$ veces, entonces $a_1+a_2=b$, luego
\begin{equation*}
    P(A_1)+P(A_2)=\frac{a_1}{b}+\frac{a_2}{b}=\frac{a_1+a_2}{b}=\frac{b}{b}=1.
\end{equation*}
Dicho de otra manera aplicando la relga de la adición, como el evento ``$A_1$ o $A_2$'' es seguro, entonces $1=P(A_1 \text{ o } A_2)=P(A_1)+P(A_2)$, pues los eventos $A_1$ y $A_2$ son mutuamente excluyentes, es decir, \emph{la suma de las probabilidades de dos eventos complementarios es la unidad}. Esta regla se puede generalizar a un número finito $A_1,A_2,\ldots,A_n$ de eventos complementarios, es decir, tales que uno y sólo uno de estos eventos necesariamente ocurre. Un conjunto de dichos eventos se llama un \emph{sistema completo de eventos}. En cualquier sistema completo de eventos se verifica que
\begin{equation*}
    P(A_1)+P(A_2)+\cdots+P(A_n) = P(A_1 \text{ o } A_2 \text{ o } \ldots \text{ o } A_n) =1
\end{equation*}
donde la primera igualdad se cumple por ser los eventos $A_1,A_2,\ldots,A_n$ mutuamente excluyentes y la segunda por ser ``$A_1$ o $A_2$ o $\ldots$ o $A_n$'' un evento seguro.

\subsection{Probabilidad condicional}

La adición a las condiciones generales en las que una operación toma lugar de una condición escencialmente nueva puede cambiar la probabilidad de un resultado de dicha operación. Al considerar información nueva en nuestros cálculos, las probabilidades pueden cambiar, incluso si la operación real no ha cambiado en absoluto. Si un resultado en una operación en masa, bajo condiciones bien definidas tiene cierta probabilidad de ocurrir, nos referiremos a esta probabilidad como \emph{probabilidad incondicional}, pero al agregar una condición adicional, la nueva probabilidad de ocurrencia de dicho evento (si es que cambia) se llamará \emph{probabilidad condicional} del evento.
\bigskip

Si $A$ y $B$ denotan eventos, denotamos por $P(A)$ a la probabilidad incondicional del evento $A$ y $P_B(A)$ a la probabilidad del evento $A$ bajo la condición de que el evento $B$ haya ocurrido. Las probabilidades incondicionales son, por su puesto, probabilidades condicionales relativas a un conjunto fijo de condiciones establecidas de antemano, pero es conveniente llamarlas así para tomar en cuenta condiciones extra.
