%%% Temario
\seclecture{Temario tentativo}{Sat 26 Jul 25}

Temario tentativo en orden cronológico:

\begin{enumerate}[label=\textnormal{(\arabic*)}]
\item Fundamentos 
\item Figura humana 
\item Manga 
\item Color 
\item Valor
\item Dibujo digital 
\item Dibujo de ropa
\item Perspectiva 
\item Diseño de personajes 
\item Storyboard
\end{enumerate}

\subsection{Algunos objetivos a mediano plazo}

Objetivos para los temas (1)--(7):

\begin{itemize}
\item Dibujar líneas rectas, curvas y círculos con confianza. 
\item Dibujar planos, cubos, cilindros, esferas en diferentes ángulos. 
\item Dibujar la figura humana con formas básicas, construcción de la figura humana.
\item Dibujar personajes de pie, de frente y de perfil con proporciones estilizadas. 
\item Dibujar personajes en vista 3/4 con estructura clara. 
\item Dibujar cabezas en vista frontal, perfil y 3/4. 
\item Dibujar expresiones básicas (feliz, triste, enojado, sorprendido).
\item Dibujar manos en vistas básicas (relajadas, puño, señalando).
\item Dibujar pies sencillos (con zapato, sin zapato). 
\item Dibujar brazos y piernas en poses naturales, sin rigidez. 
\item Dibujar prendas básicas: camisetas, pantalones, vestidos. 
\item Dibujar pliegues comunes (caída, tensión, compresión, torsión). 
\item Dibujar accesorios (cinturones, bolsos, sombreros, etc.). 
\item Aplicar ropa correctamente a figuras en distintas poses.
\item Aplicar color base limpio (sin salirte de las líneas).
\item Sombrear con cel shading en una sola fuente de luz. 
\item Aplicar luz, sombra y degradado suave si es necesario.
\item Dibujar un personaje de cuerpo completo con pose natural. 
\item Dibujar un retrato estilizado listo para una comisión. 
\item Dibujar un personaje de medio cuerpo con color y sombra simple. 
\item Recrear un personaje animado famoso con tu estilo. 
\item Simular una comisión: boceto, lineart, color, entrega.
\end{itemize}

Después, establecer objetivos para los temas (8)--(10) e integrar todo de nuevo en una comisión simulada, una \textit{request}.
